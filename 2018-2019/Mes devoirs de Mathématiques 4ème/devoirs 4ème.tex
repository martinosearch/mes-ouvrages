\documentclass[12pt,a4paper]{book}
\usepackage[minitoc]{teach}
\usepackage[utf8]{inputenc}
\usepackage[french]{babel}
\usepackage[T1]{fontenc}
\usepackage{amsmath}
\usepackage{amsfonts}
\usepackage{amssymb}
\usepackage{graphicx}
\renewcommand{\headrulewidth}{0pt}
\renewcommand{\footrulewidth}{0pt}
\fancyfoot[C]{\thepage}


\author{YAWO Kossi Atsu}
\newcommand{\prof}{YAWO Kossi Atsu}
\newcommand{\matiere}{MATHEMATIQUES}
\newcommand{\classe}{4$^{ème}$}
\title{Mes devoirs de Mathématiques}
\begin{document}
\begin{td}{\matiere}{\classe}{24 mai 2019}{\prof}
\end{td}

\begin{td}{\matiere}{\classe}{24 mai 2019}{\prof}

\end{td}

\begin{td}{\matiere}{\classe}{24 mai 2019}{\prof}

\end{td}

\begin{td}{\matiere}{\classe}{24 mai 2019}{\prof}

\end{td}

\newpage
\begin{devoir}{COMPOSITION DU TROISIÈME TRIMESTRE}{\matiere}{\classe}{2}{2H}{28 mai 2019}{\prof}
\begin{exo}[6]
\begin{enumerate}
\item Effectue le calcul suivant: $a=(\frac{4}{7}-\frac{2}{7}\times \frac{2}{3}):\frac{5}{3}$.
\item Calcule l'expression suivante et écris le résultat en notation scientifique.
\item On donne le nombre rationnel: $A=\frac{23}{6}=83,833333333 \ldots$.
\begin{enumerate}
\item A est- il aussi un nombre décimal? Justifie ta réponse.
\item Encadre A par deux nombres décimaux consécutifs d'ordre 2.
\end{enumerate}
\item Quelle est le plus grand des diviseurs communs à 108 et 180?
\end{enumerate}
\vspace{1cm}
\end{exo}

\begin{exo}[3]
\begin{enumerate}
\item Développe, réduis et ordonne:\\
$E=(x-2)^2-(x+2)(x-1)$
\item Factorise les expressions suivantes:\\
$F=36-25x^2$ et $(x-2)(x+5)+5(x+5)$
\end{enumerate}
\vspace{1cm}
\end{exo}

\begin{exo}[7]
L'unité de longueur est le km, trois villes A, B et C sont telles que: $AB=40$, $AC=30$ et $BC=50$.
\begin{enumerate}
\item Représente sur ta feuille le triangle ABC en prenant 1cm pour 10km.
\item \begin{enumerate}
\item Calcule $AB^2$; $AC^2$ et $BC^2$.
\item Déduis-en la nature du triangle ABC.
\end{enumerate}
\item On veut installer une antenne de télévision T à égale distance des trois villes A,B et C.
\begin{enumerate}
\item Montre que les trois villes A, B et C appartiennent à un même cercle $\mathcal{(C)}$ dont tu préciseras le centre.
\item Construire sur ta figure le cercle $\mathcal{(C)}$ et point T.
\end{enumerate}
\item La ville D est le symétrique de la ville A par rapport à T.
\begin{enumerate}
\item Construire sur ta figure la ville D.
\item Montre que D appartient au cercle $\mathcal{(C)}$.
\item Précise la nature du quadrilatère ABCD. Justifie.
\end{enumerate}
\end{enumerate}

\vspace{1cm}
\end{exo}

\begin{exo}[4]
\begin{enumerate}
\item On donne trois points non alignés A, B et C. Construis:
\begin{enumerate}
\item le point D tel que $\overrightarrow{AB}=\overrightarrow{DC}$.
\item le point E tel que $\overrightarrow{AC}=\overrightarrow{BE}$.
\item le point K tel que $\overrightarrow{BC}=\overrightarrow{CK}$.
\end{enumerate}
\item Réduis la somme suivantes:\\
$\overrightarrow{AB}+\overrightarrow{AC}+\overrightarrow{ED}+\overrightarrow{BF}+\overrightarrow{CE}+\overrightarrow{DA}$.
\end{enumerate}
\end{exo}
\end{devoir}

\end{document}
