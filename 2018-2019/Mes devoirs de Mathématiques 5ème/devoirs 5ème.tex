\documentclass[12pt,a4paper]{book}
\usepackage[minitoc]{teach}
\usepackage[utf8]{inputenc}
\usepackage[french]{babel}
\usepackage[T1]{fontenc}
\usepackage{amsmath}
\usepackage{amsfonts}
\usepackage{amssymb}
\usepackage{graphicx}
\renewcommand{\headrulewidth}{0pt}
\renewcommand{\footrulewidth}{0pt}
\fancyfoot[C]{\thepage}


\author{YAWO Kossi Atsu}
\newcommand{\prof}{YAWO Kossi Atsu}
\newcommand{\matiere}{MATHEMATIQUES}
\newcommand{\classe}{5$^{ème}$}
\title{Mes devoirs de Mathématiques}
\begin{document}
\begin{td}{\matiere}{\classe}{24 mai 2019}{\prof}
\end{td}

\begin{td}{\matiere}{\classe}{24 mai 2019}{\prof}

\end{td}

\begin{td}{\matiere}{\classe}{24 mai 2019}{\prof}

\end{td}

\begin{td}{\matiere}{\classe}{24 mai 2019}{\prof}

\end{td}

\newpage

\begin{td}{\matiere}{\classe}{25 mai 2019}{\prof}
\begin{exo}
\begin{enumerate}
\item Calcule: $a=8+4\times3-3$ et $b=2^5\times 5^5$
\item Dans chacun des cas suivants, trouve le nombre $m$ tel que:
\begin{enumerate}
\item $(+2)\times m=(-12)$
\item $(-3)+m=(+8)$
\end{enumerate}
\item Calcule puis simplifie si possible: $E=(\frac{7}{8}-\frac{5}{8})+\frac{1}{8}$ \qquad ;\quad $F=\frac{5}{4}-\frac{1}{6}$ \qquad et \quad $G=\frac{8}{5} \times \frac{25}{24}$
\end{enumerate}

\vspace{1cm}
\end{exo}

\begin{exo}
\begin{enumerate}
\item Un piéton fait 90 pas à la minute. Chaque pas mesure 0,80m.
\begin{enumerate}
\item Quelle est la distance parcourue en une minute?
\item Quelle distance aurait-il parcourue en 3 heures?
\end{enumerate}
\item Dans une classe de $5^{ème}$, il y a 16 filles, ce qui représente un pourcentage de 25\% du nombre d'élèves de la classe.
\begin{enumerate}
\item Quel est le nombre d'élèves de cette classe?
\item Quel est le pourcentage des garçons de cette classe?
\end{enumerate}
\end{enumerate}

\vspace{1cm}
\end{exo}

\begin{exo}
\begin{enumerate}
\item Construis un triangle ABC isocèle en A tel que $mes \widehat{A}=40\degres$.
\item Marque un point E sur le côté [AC] distinct de A et C.
\item Construis la droite $(D_1)$ parallèle à (BC) et passant par E. $(D_1)$ coupe [AB] en F.
\item Construis la droite $(D_2)$ parallèle à (AC) et passant par F. $(D_2)$ coupe [BC] en G.
\item Quelle est la nature du quadrilatère EFGC? Justifie ta réponse.
\item Calcule la mesure de chacun des angles du quadrilatère EFGC.
\item Quelle est la nature du quadrilatère EFBC? Justifie.
\end{enumerate}

\vspace{1cm}
\end{exo}
\end{td}

\newpage
\begin{devoir}{COMPOSITION DU TROISIÈME TRIMESTRE}{\matiere}{\classe}{1}{1H 30}{28 mai 2019}{\prof}
\begin{exo}[4]
\begin{enumerate}
\item Calcule: $a=8+4\times3-3$ et $b=2^5\times 5^5$
\item Dans chacun des cas suivants, trouve le nombre $m$ tel que:
\begin{enumerate}
\item $(+2)\times m=(-12)$
\item $(-3)+m=(+8)$
\end{enumerate}
\item Calcule puis simplifie si possible: $E=(\frac{7}{8}-\frac{5}{8})+\frac{1}{8}$ \qquad ;\quad $F=\frac{5}{4}-\frac{1}{6}$ \qquad et \quad $G=\frac{8}{5} \times \frac{25}{24}$
\end{enumerate}

\vspace{1cm}
\end{exo}

\begin{exo}[4]
\begin{enumerate}
\item Un piéton fait 90 pas à la minute. Chaque pas mesure 0,80m.
\begin{enumerate}
\item Quelle est la distance parcourue en une minute?
\item Quelle distance aurait-il parcourue en 3 heures?
\end{enumerate}
\item Dans une classe de $5^{ème}$, il y a 16 filles, ce qui représente un pourcentage de 25\% du nombre d'élèves de la classe.
\begin{enumerate}
\item Quel est le nombre d'élèves de cette classe?
\item Quel est le pourcentage des garçons de cette classe?
\end{enumerate}
\end{enumerate}

\vspace{1cm}
\end{exo}

\begin{exo}[8]
\begin{enumerate}
\item Construis un triangle ABC isocèle en A tel que $mes \widehat{A}=40\degres$.
\item Marque un point E sur le côté [AC] distinct de A et C.
\item Construis la droite $(D_1)$ parallèle à (BC) et passant par E. $(D_1)$ coupe [AB] en F.
\item Construis la droite $(D_2)$ parallèle à (AC) et passant par F. $(D_2)$ coupe [BC] en G.
\item Quelle est la nature du quadrilatère EFGC? Justifie ta réponse.
\item Calcule la mesure de chacun des angles du quadrilatère EFGC.
\item Quelle est la nature du quadrilatère EFBC? Justifie.
\end{enumerate}

\vspace{1cm}
\end{exo}

\begin{exo}[4]
\begin{enumerate}
\item
\begin{enumerate}
\item Rends irréductibles les fractions suivantes: $\frac{130}{490}$ et $\frac{90}{495}$
\item Déduis- en le PGCD de 130 et 490 puis le PGCD de 90 et 495.
\end{enumerate} 
\item Tu divise un nombre entier naturel $a$ par 37.
\begin{enumerate}
\item Quel est le plus petit reste possible? Calcule le nombre a dans ce cas sachant que le quotient est 23.
\item Quel est le plus grand reste possible? Calcule le nombre a dans ce cas sachant que le quotient est 23.
\end{enumerate}
\end{enumerate}
\end{exo}
\end{devoir}
\end{document}
