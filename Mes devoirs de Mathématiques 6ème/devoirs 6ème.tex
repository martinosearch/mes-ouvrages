\documentclass[12pt,a4paper]{book}
\usepackage[minitoc]{teach}
\usepackage[utf8]{inputenc}
\usepackage[french]{babel}
\usepackage[T1]{fontenc}
\usepackage{amsmath}
\usepackage{amsfonts}
\usepackage{amssymb}
\usepackage{graphicx}
\renewcommand{\headrulewidth}{0pt}
\renewcommand{\footrulewidth}{0pt}
\fancyfoot[C]{\thepage}


\author{YAWO Kossi Atsu}
\newcommand{\prof}{YAWO Kossi Atsu}
\newcommand{\matiere}{MATHEMATIQUES}
\newcommand{\classe}{6$^{ème}$}
\title{Mes devoirs de Mathématiques}
\begin{document}
\begin{td}{\matiere}{\classe}{24 mai 2019}{\prof}
\end{td}

\begin{td}{\matiere}{\classe}{24 mai 2019}{\prof}

\end{td}

\begin{td}{\matiere}{\classe}{24 mai 2019}{\prof}

\end{td}

\begin{td}{\matiere}{\classe}{24 mai 2019}{\prof}

\end{td}

\newpage
\begin{td}{\matiere}{\classe}{24 mai 2019}{\prof}
\begin{exo}
Trace un segment [AB] de longueur 4cm. Place le point I milieu du segment [AB]. Trace ensuite la droite (D) médiatrice du segment [AB]. Place sur la droite (D) le point K tel que $IK=2cm$. Trace le triangle KAB et trace le cercle $\mathcal{C}$ de centre I et de rayon [IB].
\begin{enumerate}
\item Quelle est la nature du triangle KAB? Justifie ta réponse.
\item \begin{enumerate}
\item Que représente le segment [AB] pour le cercle $\mathcal{C}$?
\item Que représente le segment [KB] pour le cercle $\mathcal{C}$?
\end{enumerate}
\begin{enumerate}
\item Calcule le périmètre du cercle $\mathcal{C}$.
\item Calcule l'aire du cercle $\mathcal{C}$.
\end{enumerate}
\end{enumerate}
\vspace{1cm}
\end{exo}

\begin{exo}
\begin{enumerate}
\item \begin{enumerate}
\item Encadre 3,2 par deux nombres entiers naturels consécutifs.
\item Encadre 1,5 par deux nombres décimaux consécutifs ayant un chiffres après la virgule.
\end{enumerate}
\item Calcule et simplifie si possible:\\
$A=\frac{25}{12}+\frac{9}{12}$ \qquad ;\qquad $B=\frac{45}{25}-\frac{10}{25}$ \qquad ;\qquad $B=3 \times\frac{11}{21}$ \qquad et\qquad $B=\frac{1}{2} \times \frac{3}{4}$
\item La TVT donne les températures suivantes pour Lomé durant une semaine:
\begin{tabular}{|c|c|c|c|c|c|c|c|}
\hline 
Jours & Lundi & Mardi & Mercredi & Jeudi & Vendredi & Samedi & Dimanche \\ 
\hline 
Température & $29\degres$ & $30\degres$ & $28\degres$ & $29\degres$ & $30\degres$ & $30\degres$ & $31\degres$ \\ 
\hline 
\end{tabular} 
\begin{enumerate}
\item Quel est le jour le plus chaud de cette semaine?
\item Quel est le jour le moins chaud de cette semaine?
\item Quel est la moyenne des températures durant cette semaine.
\end{enumerate}
\end{enumerate}
\end{exo}
\end{td}

\newpage
\begin{devoir}{COMPOSITION DU TROISIÈME TRIMESTRE}{\matiere}{\classe}{1}{1H 30}{28 mai 2019}{\prof}
\begin{exo}[4]
\begin{enumerate}
\item Effectue les opérations suivantes:\\
a) $1535,375+670$ \qquad ; \qquad b) $524,6-17,7$ \qquad et \qquad c) $49: 5$ (\emph{quotient au centième près.})
\item On donne le programme de calcul suivant: $6+4\times 10$.
\begin{enumerate}
\item Traduis ce programme par:
\begin{itemize}
\item par un schéma de calcul.
\item par une phrase en français.
\end{itemize}
\item effectue le calcul.
\end{enumerate}
\end{enumerate}
\vspace{0.3cm}
\end{exo}

\begin{exo}[6]
Trace un segment [AB] de longueur 4cm. Place le point I milieu du segment [AB]. Trace ensuite la droite (D) médiatrice du segment [AB]. Place sur la droite (D) le point K tel que $IK=2cm$. Trace le triangle KAB et trace le cercle $\mathcal{C}$ de centre I et de rayon [IB].
\begin{enumerate}
\item Quelle est la nature du triangle KAB? Justifie ta réponse.
\item \begin{enumerate}
\item Que représente le segment [AB] pour le cercle $\mathcal{C}$?
\item Que représente le segment [KB] pour le cercle $\mathcal{C}$?
\end{enumerate}
\begin{enumerate}
\item Calcule le périmètre du cercle $\mathcal{C}$.
\item Calcule l'aire du cercle $\mathcal{C}$.
\end{enumerate}
\end{enumerate}
\vspace{0.3cm}
\end{exo}

\begin{exo}[6]
\begin{enumerate}
\item \begin{enumerate}
\item Encadre 3,2 par deux nombres entiers naturels consécutifs.
\item Encadre 1,5 par deux nombres décimaux consécutifs ayant un chiffres après la virgule.
\end{enumerate}
\item Calcule et simplifie si possible:\\
$A=\frac{25}{12}+\frac{9}{12}$ \qquad ;\qquad $B=\frac{45}{25}-\frac{10}{25}$ \qquad ;\qquad $B=3 \times\frac{11}{21}$ \qquad et\qquad $B=\frac{1}{2} \times \frac{3}{4}$
\item La TVT donne les températures suivantes pour Lomé durant une semaine:
\begin{tabular}{|c|c|c|c|c|c|c|c|}
\hline 
Jours & Lundi & Mardi & Mercredi & Jeudi & Vendredi & Samedi & Dimanche \\ 
\hline 
Température & $29\degres$ & $30\degres$ & $28\degres$ & $29\degres$ & $30\degres$ & $30\degres$ & $31\degres$ \\ 
\hline 
\end{tabular} 
\begin{enumerate}
\item Quel est le jour le plus chaud de cette semaine?
\item Quel est le jour le moins chaud de cette semaine?
\item Quel est la moyenne des températures durant cette semaine.
\end{enumerate}
\end{enumerate}
\vspace{0.3cm}
\end{exo}

\begin{exo}[4]
\begin{enumerate}
\item Construis un carré ABCD de côté 3cm.
\item Construis le point E symétrique du point D par rapport à A et le point F symétrique du point D par rapport à A.
\item Quelle est nature du quadrilatère BDEF?
\end{enumerate}
\end{exo}
\end{devoir}
\end{document}
