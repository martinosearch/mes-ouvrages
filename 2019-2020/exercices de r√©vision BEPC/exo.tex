\documentclass[12pt,a4paper]{book}
\usepackage[minitoc]{teach}
\usepackage[utf8]{inputenc}
\usepackage[french]{babel}
\usepackage[T1]{fontenc}
\usepackage{amsmath}
\usepackage{amsfonts}
\usepackage{amssymb}
\usepackage{graphicx}
\renewcommand{\headrulewidth}{0pt}
\renewcommand{\footrulewidth}{0pt}


\author{YAWO Kossi Atsu}
\newcommand{\prof}{YAWO Kossi Atsu}
\newcommand{\matiere}{MATHEMATIQUES}
\newcommand{\classe}{3$^{ème}$}
\title{Mes devoirs de Mathématiques}

\begin{document}

\begin{center}
{\huge TRAVAUX DIRIGES DE MATHEMATIQUES}
\end{center}

\begin{center}
\textit{Ecole: CPL LA COURONNE D'OR}

\textit{Prof: YAWO Kossi Atsu}
\end{center}

\vspace{1cm}

\begin{exo}
On donne les polynômes suivants: \\ 
$F=(x-5)(3x+4)-(2x-10)(x-1)+x-5$ \quad et \quad $G=(x-2)^2-9$.
\begin{enumerate}
\item Développe, réduis et ordonne F et G.
\item Écris F et G sous la forme de produit de facteurs du premier degré.
\item Soit la fraction rationnelle $H=\frac{(x+1)(x-5)}{(x-5)(x+7)}$.
\begin{enumerate}
\item Détermine la condition d'existence d'une valeur numérique de H.
\item Simplifie H dans cette condition d'existence.
\item Détermine la valeur x pour $H=\frac{1}{2}$.
\item Calcule la valeur numérique de H pour $x=\sqrt{3}$.
\end{enumerate}

\end{enumerate}
\end{exo}

\vspace{0.3cm}

\begin{exo}
\begin{enumerate}
\item Calcule: $A=-\frac{5}{3}\times(4+\frac{7}{5})$ et $B=\frac{4}{5}-\frac{2}{5} \div \frac{7}{15}$
\item Soit $C=\sqrt{1053}-3\sqrt{325}+2\sqrt{52}$. Calcule C et mets le résultat sous la forme de $a\sqrt{b}$ où $b$ est le plus petit nombre entier possible.
\item Montre que $D=\frac{2-\sqrt{12}}{\sqrt{4-2\sqrt{3}}}$ est un nombre entier relatif dont tu détermineras la valeur.
\item On donne $E=\sqrt{3+2\sqrt{2}}$ et $F=\sqrt{3-2\sqrt{2}}$
\begin{enumerate}
\item Calcule $(1+\sqrt{2})^2$ et $(1-\sqrt{2})^2$.
\item Déduis-en une valeur simplifiée de E et de F.
\item Calcule: $E+F$ et $E-F$.
\end{enumerate}
\end{enumerate}
\end{exo}

\vspace{0.3cm}

\begin{exo}
Le plan étant muni d'un repère orthonormé $(O,I,J)$ et l'unité de longueur le centimètre, on considère les points: $A(-5;1)$; $B(1;7)$ et $D(1;1)$.
\begin{enumerate}
\item \begin{enumerate}
\item Place les points A, B et C dans ce repère. (On compléteras la figure au fur et à mesure).
\item Calcule les coordonnées des vecteurs $\overrightarrow{AB}$, $\overrightarrow{AD}$ et $\overrightarrow{BD}$ puis en déduire les distances $AB$, $AD$ et $BD$.
\item Quelle est la nature du triangle BAD? Justifier.
\end{enumerate}
\item On considère le point $E(7;7)$. Démontre que le quadrilatère BADE est un parallélogramme puis calculer les coordonnées de son centre M.
\item \begin{enumerate}
\item Détermine une équation de la droite $(AB)$ puis de la droite $(AE)$ sous la forme de $y=ax+b$.
\item En déduire le coefficient directeur de la droite $(DE)$.
\end{enumerate}
\item Soit $(\Delta)$ la perpendiculaire à $(AE)$ passant par $D$; $(\Delta)$ coupe  $(AE)$ en $G$.
\begin{enumerate}
\item Détermine les coordonnées du vecteur $\overrightarrow{AE}$ puis en déduire une équation de $(\Delta)$.
\item Calculer les coordonnées du point G.
\end{enumerate}
\end{enumerate}
\end{exo}
\footnote{\textit{"A person who never made a mistake never tried anything new." — Albert Einstein}}

\vspace{0.3cm}

\begin{exo}
On considère un carré $ABCD$ tel que $AB=8cm$. Soit $O$ le milieu de $[AB]$ et $E$ le point du segment $[AD]$ tel que $AE=2cm$.
\begin{enumerate}
\item Calcule $OE$, $OC$ et $CE$.
\item Démontre que $OCE$ est un triangle rectangle.
\item Soit $\alpha$ la mesure de l'angle $\widehat{AOE}$.
\begin{enumerate}
\item Calcule $cos\alpha$
\item Déduis-en l'encadrement de $\alpha$ à $1$ degré près.
\end{enumerate}
\end{enumerate}
\end{exo}

\vspace{0.3cm}

\begin{exo}
On donne les polynômes suivants: $P=(x-7)(3x+2)+x^2-49-(x+5)(x-7)$ et $Q=(x-2)^2-25$
\begin{enumerate}
\item Développe réduis et ordonne P suivant les puissances décroissantes de x.
\item Factorise P et Q.
\competence{•}{1.5}
\item Soit la fraction rationnelle: $R=\frac{(x-7)(3x+4)}{(x-7)(x+3)}$.
\begin{enumerate}
\item Détermine la condition d'existence d'une valeur numérique de R.
\item Simplifie R dans cette condition d'existence.
\item Détermine la valeur numérique de R pour $x=-2$ et pour $x=\sqrt{3}$.
\item Pour quelle valeur de $x$ a-t-on $R=0$? $R=\frac{2}{3}$?
\competence{•}{3.5}
\end{enumerate}
\end{enumerate}
\end{exo}

\vspace{0.3cm}

\begin{exo}
L'unité de longueur est le centimètre. $ABC$ est un triangle rectangle en B tel que $AB=12$ et $BC=5$. Fais une figure que tu compléteras au fur et à mesure.
\begin{enumerate}
\item  Marque le point D de $[AB]$ tel $AD=9$ puis trace la perpendiculaire à $(AB)$ en D; elle coupe $(AC)$ en $E$.
\item Calculer $AC$.
\competence{•}{1}
\item Que peux-tu dire des droite $(DE)$ et $(BC)$? Justifie.
\competence{•}{2}
\end{enumerate}
\competence{figure}{3}
\end{exo}
\footnote{\textit{"Procrastination makes easy things hard and hard things harder." — Mason Cooley}}

\vspace{0.3cm}

\begin{exo}
\begin{enumerate}
\item On donne les nombres suivants:\\
$A=(-4) \times (4-2^3)$ \qquad ; \qquad $B=\frac{(2^3)^2 \times 10^{-7}}{32 \times 10^{-8}}$ \qquad ; \qquad $C=3\sqrt{8}-3\sqrt{2}+\sqrt{32}-2\sqrt{18}$ \qquad ; \qquad $D=3\sqrt{36}+2\sqrt{100}-\sqrt{144}$ \qquad ; \qquad 
$E=\frac{3}{2}-\frac{10}{3}\times \frac{12}{5}$
\begin{enumerate}
\item Montre que A, B et D sont des nombres entiers à déterminer. 
\item Écris plus simplement C.
\item Écris E sous la forme de fractions irréductible.
\end{enumerate}
\item On donne les intervalles suivants: $A=]\leftarrow;-3[$ \qquad ; \qquad $B=]-5;2[$ \qquad et \qquad $C=]1;7[$
\begin{enumerate}
\item Traduis chacun de ces intervalles par une inégalité.
\item Détermine: $A\cap B$ \qquad ; \qquad $B\cup C$ \qquad et \qquad $A \cap C$
\end{enumerate}
\end{enumerate}
\end{exo}

\vspace{0.3cm}
\begin{exo}[5]
On donne les polynômes suivants:\\
$M=4(x-1)^2-(x-5)^2$ \qquad ; \qquad $N=x^2-6x+9-(3-x)(2x+1)$
\begin{enumerate}
\item Développe, réduis et ordonne M suivant les puissances décroissantes de x.
\item Écris M et N sous la forme de produit de facteurs du premier degré.
\item On considère la rationnelle $H=\frac{x^2-6x+9}{(3x-2)(x-3)}$
\begin{enumerate}
\item Détermine la condition d'existence d'une valeur numérique de H.
\item Simplifie H lorsqu'elle existe.
\item Calcule la valeur numérique de H pour $x=\sqrt{2}$, écris le résultat sans radical au dénominateur.
\end{enumerate}
\end{enumerate}
\end{exo}

\vspace{0.3cm}

\begin{exo}[5]
L'unité de longueur est le centimètre. Soit $\mathcal{(C)}$ le demi-cercle de diamètre $[NI]$ tel que $NI=10$. O est un point de $\mathcal{(C)}$ tel que $OI=6$.
\begin{enumerate}
\item Fais une figure que tu complèteras au fur et à mesure.
\item Démontre que le triangle $NIO$ est rectangle.
\item Calcule NO.
\item H est le projeté orthogonal de $O$ sur $[NI]$. Calcule OH, $tan\widehat{INO}$ puis déduis un encadrement d'ordre zéro de la mesure de l'angle $\widehat{INO}$.
\item Place le point P sur le segment $[NO]$ tel que $\overrightarrow{OP}=\frac{3}{4}\overrightarrow{ON}$. La parallèle à $(OI)$ passant par $P$ coupe $[NI]$ en $R$. Calcule $PR$.
\end{enumerate}
\end{exo}

\vspace{0.3cm}

\begin{exo}
Dans le plan muni d'un repère orthonormé (O,I,J) on donne les points: $A(-3;0)$ ; $B(2;-3)$ et $C(5;2)$.
\begin{enumerate}
\item Calcule AB, BC et AC.
\item Justifie que le triangle ABC est rectangle et isocèle.
\item Calcule les coordonnées du point K milieu de [AC].
\item D est l'image de B par la symétrie de centre K. Calcule les coordonnées de D.
\item Donne et justifie la nature du quadrilatère ABCD.
\end{enumerate}
\end{exo}

\vspace{0.3cm}

\begin{exo}
Dans le plan muni d'un repère orthonormé (O, I, J), on considère les points A, B et C tel que:\\
$\overrightarrow{OA}=7\overrightarrow{OI}+\overrightarrow{OJ}$ \quad ; \quad $\overrightarrow{OB}=8\overrightarrow{OI}+4\overrightarrow{OJ}$ \quad et \quad $\overrightarrow{CO}=\overrightarrow{OI}-7\overrightarrow{OJ}$.
\begin{enumerate}
\item Place les points A, B et C dans le repère.
\item \begin{enumerate}
\item Montre que les vecteurs $\overrightarrow{AB}$ et $\overrightarrow{BC}$ sont orthogonaux.
\item Donne en justifie la nature du triangle ABC.
\end{enumerate}
\item Soient le point M milieu du segment [AC] et le point D symétrique de B par rapport à M.
\begin{enumerate}
\item Détermine les coordonnées de M et de D.
\item Précise la nature du quadrilatère ABCD. Justifie.
\end{enumerate} 
\item \begin{enumerate}
\item Construis le cercle $\mathcal{(C)}$ circonscrit au quadrilatère ABCD.
\item Précise son centre, calcule son rayon et montre qu'il passe par le point O.
\end{enumerate}
\end{enumerate}
\end{exo}
\footnote{\textit{"The harder you work for something, the greater you'll feel when you achieve it."}}


\end{document}
