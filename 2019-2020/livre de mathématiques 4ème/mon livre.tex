\documentclass[nocrop]{sesamanuel}
\usepackage{sesamanuelTIKZ}
\let\ifluatex\relax
\usepackage{hyperref}


%complement
\usepackage{tikz}  %   appel  général
\usepackage{pgf}  %   appel  général
\usepackage{tkz-tab}  %   extention  pour  les  tableaux  de  variations
\usetikzlibrary{arrows}  %   librairie  pour  les  flèches
\usetikzlibrary{patterns}  %   librairie  pour  les  hachures  et  les  pointillés

\themaG
\begin{document}
\chapter{Les distances}

\begin{acquis}
\begin{itemize}
\item  Premier  point  à  connaître.
\item  Autre  point  à  savoir  faire.
\item  Dernier  point  devant  être  su.
\end{itemize}
\end{acquis}

\cours
\section{Distances de deux points}
\begin{definition}
La distance des points $A$ et $B$ est la longueur du segment [AB].
\begin{exemple}
\vspace{0.5cm}
\begin{tikzpicture}[general]
\coordinate  (A)  at  (0,0);
\coordinate  (B)  at  (4,0);
\draw  (A)--(B);
\pointC{A}{A}{below  left}
\pointC{B}{B}{below  right}
\end{tikzpicture}


$AB=5cm$
\end{exemple}
\end{definition}

\section{Distances d'un point à une droite}
\begin{definition}
On considère une droite $(D)$ et un point A n'appartenant pas à la droite $(D)$.\\
\begin{itemize}
\item On appelle \emph{projection orthogonale} du point A sur la droite $(D)$, la perpendiculaire à $(D)$ passant par A.
\item Si H est le point d'intersection de (D) et de la projection orthogonale de A sur $(D)$, alors le point H est appelé, le \emph{projeté orthogonal} du point A sur la droite $(D)$.
\end{itemize}
\end{definition}

\begin{definition}
La distance du point A à la droite (D) est la distance du point A et du projeté orthogonal de A sur $(D)$
\end{definition}

\section{Distance de deux droites parallèles}

\section{Points équidistants de deux droites parallèles}

\section{Points équidistants de deux droites sécantes}

\connaissances

\QCMautoevaluation{texte  introductif}
\begin{QCM}

\begin{EnonceCommunQCM}
Pour  les  questions  \RefQCM{premier-qcm}  à \RefQCM{deuxieme-qcm},  $f$  désigne  une fonction  affine.
\end{EnonceCommunQCM}

\begin{GroupeQCM}

\begin{exercice} \label{premier-qcm} La  courbe  de  $f$  est
\begin{ChoixQCM}{3}
\item  une  droite
\item  une  parabole
\item  autre
\end{ChoixQCM}
\end{exercice}

\begin{corrige}
\reponseQCM{a}
\end{corrige}

\begin{exercice} \label{deuxieme-qcm}
$f(3)$
\begin{ChoixQCM}{3}
\item  vaut  la  moitié  de  $f(6)$
\item  vaut  le  double  de  $f(6)$
\item  on  ne  peut  pas savoir
\end{ChoixQCM}

\end{exercice}

\begin{corrige}
\reponseQCM{c}
\end{corrige}

\end{GroupeQCM}
\end{QCM}



\exercicesbase
\begin{colonne*exercice}
\begin{exercice}
On donne un rectangle ABCD tel que AB=6cm et BC=4cm.
\begin{enumerate}
\item Quelle est la distance du point A à la droite (CD)?
\item Quelle est la distance du point D à la droite (BC)?
\end{enumerate}
\end{exercice}

\begin{exercice}
On donne un carré EFGH de côté 5cm de centre O.
\begin{enumerate}
\item Quelle est la distance du point E à la droite (FG)?
\item Mesure la distance du point G à la droite (GF)?
\item Mesure la distance du point O à la droite (HG)?
\end{enumerate}
\end{exercice}
\end{colonne*exercice}

\exercicesappr


\chapter{Le théorème de Pythagore}
\cours
\section{Le théorème direct}
\section{La réciproque du théorème de Pythagore}

\exercicesbase
\begin{colonne*exercice}
\begin{exercice}
On donne un rectangle ABCD tel que AB=6cm et BC=4cm.
\begin{enumerate}
\item Quelle est la distance du point A à la droite (CD)?
\item Quelle est la distance du point D à la droite (BC)?
\end{enumerate}
\end{exercice}

\begin{exercice}
On donne un carré EFGH de côté 5cm de centre O.
\begin{enumerate}
\item Quelle est la distance du point E à la droite (FG)?
\item Mesure la distance du point G à la droite (GF)?
\item Mesure la distance du point O à la droite (HG)?
\end{enumerate}
\end{exercice}
\begin{exercice}
On donne un rectangle ABCD tel que AB=6cm et BC=4cm.
\begin{enumerate}
\item Quelle est la distance du point A à la droite (CD)?
\item Quelle est la distance du point D à la droite (BC)?
\end{enumerate}
\end{exercice}

\begin{exercice}
On donne un carré EFGH de côté 5cm de centre O.
\begin{enumerate}
\item Quelle est la distance du point E à la droite (FG)?
\item Mesure la distance du point G à la droite (GF)?
\item Mesure la distance du point O à la droite (HG)?
\end{enumerate}
\end{exercice}

\begin{exercice}
On donne un rectangle ABCD tel que AB=6cm et BC=4cm.
\begin{enumerate}
\item Quelle est la distance du point A à la droite (CD)?
\item Quelle est la distance du point D à la droite (BC)?
\end{enumerate}
\end{exercice}

\begin{exercice}
On donne un carré EFGH de côté 5cm de centre O.
\begin{enumerate}
\item Quelle est la distance du point E à la droite (FG)?
\item Mesure la distance du point G à la droite (GF)?
\item Mesure la distance du point O à la droite (HG)?
\end{enumerate}
\end{exercice}

\begin{exercice}
On donne un rectangle ABCD tel que AB=6cm et BC=4cm.
\begin{enumerate}
\item Quelle est la distance du point A à la droite (CD)?
\item Quelle est la distance du point D à la droite (BC)?
\end{enumerate}
\end{exercice}

\begin{exercice}
On donne un carré EFGH de côté 5cm de centre O.
\begin{enumerate}
\item Quelle est la distance du point E à la droite (FG)?
\item Mesure la distance du point G à la droite (GF)?
\item Mesure la distance du point O à la droite (HG)?
\end{enumerate}
\end{exercice}

\begin{exercice}
On donne un rectangle ABCD tel que AB=6cm et BC=4cm.
\begin{enumerate}
\item Quelle est la distance du point A à la droite (CD)?
\item Quelle est la distance du point D à la droite (BC)?
\end{enumerate}
\end{exercice}

\begin{exercice}
On donne un carré EFGH de côté 5cm de centre O.
\begin{enumerate}
\item Quelle est la distance du point E à la droite (FG)?
\item Mesure la distance du point G à la droite (GF)?
\item Mesure la distance du point O à la droite (HG)?
\end{enumerate}
\end{exercice}

\begin{exercice}
On donne un rectangle ABCD tel que AB=6cm et BC=4cm.
\begin{enumerate}
\item Quelle est la distance du point A à la droite (CD)?
\item Quelle est la distance du point D à la droite (BC)?
\end{enumerate}
\end{exercice}

\begin{exercice}
On donne un carré EFGH de côté 5cm de centre O.
\begin{enumerate}
\item Quelle est la distance du point E à la droite (FG)?
\item Mesure la distance du point G à la droite (GF)?
\item Mesure la distance du point O à la droite (HG)?
\end{enumerate}
\end{exercice}
\end{colonne*exercice}

\exercicesappr
\begin{colonne*exercice}
\begin{exercice}
On donne un rectangle ABCD tel que AB=6cm et BC=4cm.
\begin{enumerate}
\item Quelle est la distance du point A à la droite (CD)?
\item Quelle est la distance du point D à la droite (BC)?
\end{enumerate}
\end{exercice}

\begin{exercice}
On donne un carré EFGH de côté 5cm de centre O.
\begin{enumerate}
\item Quelle est la distance du point E à la droite (FG)?
\item Mesure la distance du point G à la droite (GF)?
\item Mesure la distance du point O à la droite (HG)?
\end{enumerate}
\end{exercice}
\end{colonne*exercice}


\chapter{Le théorème de la droite des milieux}
\section{$1^{ere}$ Propriété}
dsfsdsfg
\section{$2^{e}$ Propriété}
\section{$3^{e}$ Propriété}

\end{document}
