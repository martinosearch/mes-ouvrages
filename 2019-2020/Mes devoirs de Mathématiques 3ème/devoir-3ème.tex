\documentclass[12pt,a4paper]{book}
\usepackage[minitoc]{teach}
\usepackage[utf8]{inputenc}
\usepackage[french]{babel}
\usepackage[T1]{fontenc}
\usepackage{amsmath}
\usepackage{amsfonts}
\usepackage{amssymb}
\usepackage{graphicx}
\renewcommand{\headrulewidth}{0pt}
\renewcommand{\footrulewidth}{0pt}


\author{YAWO Kossi Atsu}
\newcommand{\prof}{YAWO Kossi Atsu}
\newcommand{\matiere}{MATHEMATIQUES}
\newcommand{\classe}{3$^{ème}$}
\title{Mes devoirs de Mathématiques}
\begin{document}
\begin{devoir}{DEVOIR SURVEILLE DU PREMIER TRIMESTRE}{\matiere}{\classe}{2}{2H}{13 novembre 2019}{\prof}
\begin{exo}[5]
\begin{enumerate}
\item Calcule et mets les résultats sous la forme de fractions irréductibles:
$A=\frac{5}{3}-\frac{1}{3}\times\frac{9}{16}$ \qquad ; \qquad $B=\frac{3}{4}-\frac{2}{3}:\frac{18}{15}$.
\item Calcule et écris le résultat en notation scientifique: $C=\frac{8\times10^8\times1,6}{0,4\times10^{-3}}$.
\competence{•}{3}
\item Les dimensions d'un terrain rectangulaire sont proportionnelles à 5 et 9. Sachant que le demi-périmètre de ce terrain est 280m,
\begin{enumerate}
\item Détermine la longueur et la largeur de ce terrain.
\item Calcule l'aire de ce terrain.
\competence{•}{2}
\end{enumerate}
\end{enumerate}
\end{exo}

\vspace{1cm}
\begin{exo}[4]
\begin{enumerate}
\item On pose $D=\sqrt{3+2\sqrt{2}}$ et $E=\sqrt{3-2\sqrt{2}}$.
\begin{enumerate}
\item Calcule $(1+\sqrt{2})^2$ et $(1-\sqrt{2})^2$
\item En déduire la valeur la plus simple de $D$ et $E$.
\item Calcule $D+E$ ; $D-E$ et $\frac{D}{E}$.
\end{enumerate}
\item Ecris sous la forme de $a\sqrt{b}$: $C=\sqrt{12}-\sqrt{3}+\sqrt{48}$
\item Montre que $C=(\sqrt{5}+\sqrt{10})^2-10\sqrt{2}$ est un nombre entier.
\end{enumerate}
\competence{•}{4}
\end{exo}

\vspace{1cm}

\begin{exo}[5]
On donne les polynômes suivants: $P=(x-7)(3x+2)+x^2-49-(x+5)(x-7)$ et $Q=(x-2)^2-25$
\begin{enumerate}
\item Développe réduis et ordonne P suivant les puissances décroissantes de x.
\item Factorise P et Q.
\competence{•}{1.5}
\item Soit la fraction rationnelle: $R=\frac{(x-7)(3x+4)}{(x-7)(x+3)}$.
\begin{enumerate}
\item Détermine la condition d'existence d'une valeur numérique de R.
\item Simplifie R dans cette condition d'existence.
\item Détermine la valeur numérique de R pour $x=-2$ et pour $x=\sqrt{3}$.
\item Pour quelle valeur de $x$ a-t-on $R=0$? $R=\frac{2}{3}$?
\competence{•}{3.5}
\end{enumerate}
\end{enumerate}
\end{exo}

\vspace{1cm}


\begin{exo}[6]
L'unité de longueur est le centimètre. $ABC$ est un triangle rectangle en B tel que $AB=12$ et $BC=5$. Fais une figure que tu compléteras au fur et à mesure.
\begin{enumerate}
\item  Marque le point D de $[AB]$ tel $AD=9$ puis trace la perpendiculaire à $(AB)$ en D; elle coupe $(AC)$ en $E$.
\item Calculer $AC$.
\competence{•}{1}
\item Que peux-tu dire des droite $(DE)$ et $(BC)$? Justifie.
\competence{•}{2}
\end{enumerate}
\competence{figure}{3}
\end{exo}
\tableofcompetences
\end{devoir}

\newpage
\begin{td}{\matiere}{\classe}{23 novembre 2019}{\prof}
\begin{exo}
On donne les polynômes suivants: $M=1-16x^2$ et $N=(3x+15)(x+2)+x^2-25$.
\begin{enumerate}
\item Développe, réduis et ordonne suivants les puissances de x le polynôme N.
\item Écris $M$ et $N$ sous la forme d'un produit de facteurs du premier degré.
\item Soit la fraction rationnelle: $F=\frac{1-16x^2}{(x+5)(4x+1)}$
\begin{enumerate}
\item Détermine la condition d'existence d'une valeur numérique de $F$.
\item Simplifie l'expression de $F$ lorsqu'elle existe.
\item Détermine la valeur numérique de $F$ pour $x=-2$ et pour $x=\sqrt{3}$.
\item Pour quelle valeur de x a-t-on $F=-\frac{2}{3}$
\end{enumerate}
\end{enumerate}
\end{exo}
\vspace{1cm}

\begin{exo}
\begin{enumerate}
\item On donne : $A=\frac{8}{3}+(\frac{3}{4}-\frac{5}{6})+\frac{3}{16}$.\\
Calcule A et donne le résultat sous forme de fraction irréductible.
\item Calcule $(3-2\sqrt{5})^2$ et écris plus simplement $B=\sqrt{29-12\sqrt{5}}$.
\item On considère l'expression $C=\frac{\sqrt{2}(\sqrt{2}-\sqrt{3})}{\sqrt{3}+\sqrt{2}}$
\begin{enumerate}
\item Rend rationnel le dénominateur de $A$.
\item Donne un encadrement de $A$ à $10^{-2}$ près sachant que $1,414<\sqrt{2}<1,415$ et $1,732<\sqrt{3}<1,733$
\end{enumerate}
\end{enumerate}
\end{exo}
\vspace{1cm}

\begin{exo}
$\mathcal{(C)}$ est un cercle de centre $O$ et de diamètre $[BC]$ tel que $BC=8$. $A$ est un point du cercle $\mathcal{(C)}$ tel que $BA=4$. $B'$ est le symétrique de B par rapport à $A$. 
\begin{enumerate}
\item Démontre que le triangle $BAC$ est rectangle en $A$.
\item Calcule $AC$.
\item Démontre que $AOB$ est un triangle équilatéral.
\item Calcule la mesure en degré de chacun des angles du triangle $AOC$.
\item Calcule $BB'$.
\item Démontre que $(AC)$ est la médiatrice de $[BB']$
\end{enumerate}
\end{exo}
\end{td}

\newpage
\begin{devoir}{COMPOSITION DU PREMIER TRIMESTRE}{\matiere}{\classe}{2}{2H}{13 décembre 2019}{\prof}

\begin{exo}[4]
\begin{enumerate}
\item Calcule et donne le résultat sous forme de fraction irréductible: $A=\frac{8}{3}+(\frac{3}{4}-\frac{5}{6})+\frac{3}{16}$ \qquad ;\qquad $B=\frac{5}{3}-\frac{1}{3}\times\frac{9}{16}$
\competence{fraction}{1.5}
\item Calcule et donne le résultat en notation scientifique: $C=\frac{6\times 10^5-6\times10^3}{3\times10^{11}}$
\competence{puissance}{1}
\item Calcule $(3-2\sqrt{5})^2$ et écris plus simplement $D=\sqrt{29-12\sqrt{5}}$.
\item Ecris sous la forme de $a\sqrt{b}$ où $a$ et $b$ sont des nombres entiers et $b$ le plus petit possible: $E=\sqrt{27}+7\sqrt{75}-\sqrt{300}$
\competence{racine carrées}{1.5}
\end{enumerate}
\end{exo}
\vspace{0.5cm}

\totalexo



\begin{exo}[5]
On donne les polynômes suivants: $M=1-16x^2$ et $N=(3x+15)(x+2)+x^2-25$.
\begin{enumerate}
\item Développe, réduis et ordonne suivants les puissances de x le polynôme N.
\competence{développer}{0.5}
\item Écris $M$ et $N$ sous la forme d'un produit de facteurs du premier degré.
\competence{factoriser}{1}
\item Soit la fraction rationnelle: $F=\frac{1-16x^2}{(x+5)(4x+1)}$
\begin{enumerate}
\item Détermine la condition d'existence d'une valeur numérique de $F$.
\competence{condition d'existence}{0.75}
\item Simplifie l'expression de $F$ lorsqu'elle existe.
\competence{simplifier}{0.75}
\item Détermine la valeur numérique de $F$ pour $x=-2$ et pour $x=\sqrt{3}$.
\competence{valeur numérique}{1.5}
\item Pour quelle valeur de x a-t-on $F=-\frac{2}{3}$
\competence{équation}{0.5}
\end{enumerate}
\end{enumerate}
\end{exo}
\vspace{0.5cm}

\totalexo
\begin{exo}[6]
\begin{enumerate}
\item Construire un triangle ABC tel que $AC=12cm$, $AB=13cm$ et $BC=5cm$.
\item Placer le point R appartenant à $[AC]$ tel que $AR=9cm$.
\item Placer le point T appartenant à $[AB]$ tel que la droite $(RT)$ soit perpendiculaire à la droite $(AC)$.
\item Démontre que le triangle $ABC$ est rectangle.
\item Que peut-on dire des droites $(RT)$ et $(BC)$? Justifier.
\item Calcule la valeur exacte de la longueur du segment $[AT]$.
\end{enumerate}

\competence{•}{6}
\end{exo}

\totalexo

\vspace{0.5cm}

\begin{exo}[5]
$\mathcal{(C)}$ est un cercle de centre $O$ et de diamètre $[BC]$ tel que $BC=8$. $A$ est un point du cercle $\mathcal{(C)}$ tel que $BA=4$. $B'$ est le symétrique de B par rapport à $A$. 
\begin{enumerate}
\item Démontre que le triangle $BAC$ est rectangle en $A$.
\competence{cercle circonscrit à un triangle}{0.5}
\item Calcule $AC$.
\competence{théorème de Pythagore}{1}
\item Démontre que $AOB$ est un triangle équilatéral.
\competence{cercle et distance}{1}
\item Calcule la mesure en degré de chacun des angles du triangle $AOC$.
\competence{angles d'un triangle}{1.5}
\item Calcule $BB'$.
\competence{symétrie}{0.5}
\item Démontre que $(AC)$ est la médiatrice de $[BB']$
\competence{médiatrice d'un segment}{0.5}
\end{enumerate}
\end{exo}

\totalexo

\tableofcompetences
\end{devoir}

\newpage
\begin{td}{\matiere}{\classe}{25 janvier 2020}{\prof}
\begin{exo}
\begin{enumerate}
\item On donne les polynômes suivants: \\
$F=(12x^2-3)(x+3)+(x^2-9)(2x-1)$ \qquad et \qquad $G=4x^3-x$.
\begin{enumerate}
\item Factorise F et G.
\item Développe, réduis et ordonne F suivant les puissances croissantes de $x$.
\end{enumerate}
\item Soit H la fraction rationnelle telle que: $H=\frac{(12x^2-3)(x+3)+(x^2-9)(2x-1)}{4x^3-x}$.
\begin{enumerate}
\item Trouve la condition d'existence d'une valeur numérique de H.
\item Simplifie l'écriture de H.
\item Pour quelle valeur numérique de H=0? H=1.
\item Calcule la valeur numérique de H pour $x=\sqrt{2}$ puis donne un encadrement à $10^-2$ près de cette valeur sachant que: $1,414<\sqrt{2}<1,415$.
\end{enumerate}
\end{enumerate}
\end{exo}

\vspace{1cm}
\begin{exo}
Le plan étant muni d'un repère orthonormé $(O,I,J)$ et l'unité de longueur le centimètre, on considère les points: $A(-5;1)$; $B(1;7)$ et $D(1;1)$.
\begin{enumerate}
\item \begin{enumerate}
\item Place les points A, B et C dans ce repère. (On compléteras la figure au fur et à mesure).
\item Calcule les coordonnées des vecteurs $\overrightarrow{AB}$, $\overrightarrow{AD}$ et $\overrightarrow{BD}$ puis en déduire les distances $AB$, $AD$ et $BD$.
\item Quelle est la nature du triangle BAD? Justifier.
\end{enumerate}
\item On considère le point $E(7;7)$. Démontre que le quadrilatère BADE est un parallélogramme puis calculer les coordonnées de son centre M.
\item \begin{enumerate}
\item Détermine une équation de la droite $(AB)$ puis de la droite $(AE)$ sous la forme de $y=ax+b$.
\item En déduire le coefficient directeur de la droite $(DE)$.
\end{enumerate}
\item Soit $(\Delta)$ la perpendiculaire à $(AE)$ passant par $D$; $(\Delta)$ coupe  $(AE)$ en $G$.
\begin{enumerate}
\item Détermine les coordonnées du vecteur $\overrightarrow{AE}$ puis en déduire une équation de $(\Delta)$.
\item Calculer les coordonnées du point G.
\end{enumerate}
\end{enumerate}
\end{exo}
\end{td}
\footnote{\textit{"A person who never made a mistake never tried anything new." — Albert Einstein}}
\footnote{\textit{"Procrastination makes easy things hard and hard things harder." — Mason Cooley}}

\newpage
\begin{devoir}{DEVOIR SURVEILLÉ DU DEUXIÈME TRIMESTRE}{\matiere}{\classe}{2}{2H}{12 février 2020}{\prof}
\begin{exo}[5,5pts]
\begin{enumerate}
\item On donne les nombres suivants:\\
$A=(-4) \times (4-2^3)$ \qquad ; \qquad $B=\frac{(2^3)^2 \times 10^{-7}}{32 \times 10^{-8}}$ \qquad ; \qquad $C=3\sqrt{8}-3\sqrt{2}+\sqrt{32}-2\sqrt{18}$ \qquad ; \qquad $D=3\sqrt{36}+2\sqrt{100}-\sqrt{144}$ \qquad ; \qquad 
$E=\frac{3}{2}-\frac{10}{3}\times \frac{12}{5}$
\begin{enumerate}
\item Montre que A, B et D sont des nombres entiers à déterminer. 
\item Écris plus simplement C.
\item Écris E sous la forme de fractions irréductible.
\end{enumerate}
\item On donne les intervalles suivants: $A=]\leftarrow;-3[$ \qquad ; \qquad $B=]-5;2[$ \qquad et \qquad $C=]1;7[$
\begin{enumerate}
\item Traduis chacun de ces intervalles par une inégalité.
\item Détermine: $A\cap B$ \qquad ; \qquad $B\cup C$ \qquad et \qquad $A \cap C$
\end{enumerate}
\end{enumerate}
\end{exo}

\vspace{0.5cm}
\begin{exo}[5]
On donne les polynômes suivants:\\
$M=4(x-1)^2-(x-5)^2$ \qquad ; \qquad $N=x^2-6x+9-(3-x)(2x+1)$
\begin{enumerate}
\item Développe, réduis et ordonne M suivant les puissances décroissantes de x.
\item Écris M et N sous la forme de produit de facteurs du premier degré.
\item On considère la rationnelle $H=\frac{x^2-6x+9}{(3x-2)(x-3)}$
\begin{enumerate}
\item Détermine la condition d'existence d'une valeur numérique de H.
\item Simplifie H lorsqu'elle existe.
\item Calcule la valeur numérique de H pour $x=\sqrt{2}$, écris le résultat sans radical au dénominateur.
\end{enumerate}
\end{enumerate}
\end{exo}

\vspace{0.5cm}

\begin{exo}[5]
L'unité de longueur est le centimètre. Soit $\mathcal{(C)}$ le demi-cercle de diamètre $[NI]$ tel que $NI=10$. O est un point de $\mathcal{(C)}$ tel que $OI=6$.
\begin{enumerate}
\item Fais une figure que tu complèteras au fur et à mesure.
\item Démontre que le triangle $NIO$ est rectangle.
\item Calcule NO.
\item H est le projeté orthogonal de $O$ sur $[NI]$. Calcule OH, $tan\widehat{INO}$ puis déduis un encadrement d'ordre zéro de la mesure de l'angle $\widehat{INO}$.
\item Place le point P sur le segment $[NO]$ tel que $\overrightarrow{OP}=\frac{3}{4}\overrightarrow{ON}$. La parallèle à $(OI)$ passant par $P$ coupe $[NI]$ en $R$. Calcule $PR$.
\end{enumerate}
\end{exo}

\begin{exo}[4,5]
Dans le plan muni d'un repère orthonormé (O,I,J) on donne les points: $A(-3;0)$ ; $B(2;-3)$ et $C(5;2)$.
\begin{enumerate}
\item Calcule AB, BC et AC.
\item Justifie que le triangle ABC est rectangle et isocèle.
\item Calcule les coordonnées du point K milieu de [AC].
\item D est l'image de B par la symétrie de centre K. Calcule les coordonnées de D.
\item Donne et justifie la nature du quadrilatère ABCD.
\end{enumerate}
\end{exo}
\tableofcompetences
\end{devoir}


\newpage


\begin{td}{\matiere}{\classe}{29 février 2020}{\prof}
\begin{exo}
Dans le plan muni d'un repère orthonormé (O, I, J), on considère les points A, B et C tel que: $\overrightarrow{OA}=7\overrightarrow{OI}+\overrightarrow{OJ}$ ; $\overrightarrow{OB}=8\overrightarrow{OI}+4\overrightarrow{OJ}$ et $\overrightarrow{CO}=\overrightarrow{OI}-7\overrightarrow{OJ}$.
\begin{enumerate}
\item Place les points A, B et C dans le repère.
\item \begin{enumerate}
\item Montre que les vecteurs $\overrightarrow{AB}$ et $\overrightarrow{BC}$ sont orthogonaux.
\item Donne en justifie la nature du triangle ABC.
\end{enumerate}
\item Soient le point M milieu du segment [AC] et le point D symétrique de B par rapport à M.
\begin{enumerate}
\item Détermine les coordonnées de M et de D.
\item Précise la nature du quadrilatère ABCD. Justifie.
\end{enumerate} 
\item \begin{enumerate}
\item Construis le cercle $\mathcal{(C)}$ circonscrit au quadrilatère ABCD.
\item Précise son centre, calcule son rayon et montre qu'il passe par le point O.
\end{enumerate}
\end{enumerate}
\end{exo}

\vspace{1cm}

\begin{exo}
Le plan est muni d'un repère orthonormé (O, I, J). L'unité de longueur est le centimètre.
\begin{enumerate}
\item \begin{enumerate}
\item Place dans le repère les points A(0;4), B(6;1) et C(2;-3).
\item Construis H le projeté orthogonal du point C sur la droite (AB).
\end{enumerate}
\item On se propose de déterminer les coordonnées du point H; pour cela:
\begin{enumerate}
\item Détermine une équation cartésienne de la droite (AB).
\item Détermine le coefficient directeur et une équation cartésienne de la droite (CH).
\item Déduis- en les coordonnées de H.
\end{enumerate}
\end{enumerate}

\vspace{0.3cm}
\end{exo}

\vspace{1cm}

\begin{exo}
Soient $f$ et $g$ deux applications définies sur $\mathbb{R}$ par:\\
$f(x)=(9x^2-25)(4x-1)+(16x^2-8x+1)(6x-10)$ et $g(x)=(3x-5)[(5x-1)^2-4(3x+2)^2]$
\begin{enumerate}
\item Mettre $f(x)$ et $g(x)$ sous la forme de produit de facteurs du premier degré.
\item On pose $Q(x)=\frac{f(x)}{g(x)}$\\
Quelle est la condition d'existence d'une valeur numérique de Q? Simplifie Q.
\item On définit dans $\mathbb{R}$ la fraction rationnelle $S(x)=\frac{1-4x}{x+5}$
\begin{enumerate}
\item Calcule $S(\sqrt{3})$ et rend rationnelle le dénominateur.
\item Résoudre dans $\mathbb{R}$ l'équation $S(x)=1$
\end{enumerate}
\end{enumerate}
\end{exo}

\end{td}

\newpage

\begin{devoir}{COMPOSITION DU DEUXIÈME TRIMESTRE}{\matiere}{\classe}{2}{2H}{11 mars 2020}{\prof}

\begin{exo}[4]
\begin{enumerate}
\item Calcule les nombres: $M=\frac{2}{7}-\frac{3}{7}\times(\frac{2}{3})^2$ \quad et \quad $N=(\frac{2}{3}-3)\div \frac{1}{9}+7$
\item On considère les nombres : $A=\sqrt{81}-\sqrt{108}+\sqrt{48}-\sqrt{25}$ \quad ; \quad $B=(1-\sqrt{3})^2$ \quad et \quad $C=\sqrt{A}$.
\begin{enumerate}
\item Calcule B et montre que: $A=4-2\sqrt{3}$.
\item Déduis-en une écriture simplifiée de C.
\end{enumerate} 
\end{enumerate}
\end{exo}

\vspace{0.5cm}

\begin{exo}[5]
\begin{enumerate}
\item On donne les expressions littérales suivantes: $E=2x(3x-4)-(4-3x)^2$ et $F=(9x^2-4)-(2x-3)(3x-2)$
\begin{enumerate}
\item Développe, réduis et ordonne E suivant les puissances croissantes de x.
\item Factorise E et F.
\item Résous dans R l'équation $(3x-2)(x+5)=0$.
\end{enumerate}
\item Soit Q la fraction rationnelle telle que $Q=\frac{3x(x-3)-2(x-3)}{(3x-2)(x+5)}$
\begin{enumerate}
\item Détermine la condition d'existence d'une valeur numérique de Q.
\item Simplifie Q dans cette condition.
\item Calcule la valeur exacte de Q pour $x=\sqrt{2}$.
\end{enumerate}
\end{enumerate}
\end{exo}

\vspace{0.5cm}

\begin{exo}[4]
L'unité de longueur est le centimètre. $ABC$ est un triangle rectangle en B tel que $AB=6$ et $AC=8$. M et N sont deux points tels que $\overrightarrow{AM}=\frac{1}{2}\overrightarrow{AB}$ et $\overrightarrow{AN}=\frac{1}{2}\overrightarrow{AC}$.
\begin{enumerate}
\item Trouve les distances AM et AN puis fais la figure.
\item Justifie que les droites (BC) et (MN) sont parallèles.
\item Calcule $BC$, $cos\widehat{BAC}$ et $tan\widehat{BAC}$.
\item A l'aide de la table trigonométrique, trouve l'encadrement d'ordre zéro de la mesure de l'angle $\widehat{BAC}$.
\end{enumerate}
\end{exo}

\vspace{0.5cm}

\begin{exo}[7]
Dans le plan muni d'un repère orthonormé (O, I, J), on considère les points A, B et C tel que:\\
$\overrightarrow{OA}=7\overrightarrow{OI}+\overrightarrow{OJ}$ \quad ; \quad $\overrightarrow{OB}=8\overrightarrow{OI}+4\overrightarrow{OJ}$ \quad et \quad $\overrightarrow{CO}=\overrightarrow{OI}-7\overrightarrow{OJ}$.
\begin{enumerate}
\item Place les points A, B et C dans le repère.
\item \begin{enumerate}
\item Montre que les vecteurs $\overrightarrow{AB}$ et $\overrightarrow{BC}$ sont orthogonaux.
\item Donne en justifie la nature du triangle ABC.
\end{enumerate}
\item Soient le point M milieu du segment [AC] et le point D symétrique de B par rapport à M.
\begin{enumerate}
\item Détermine les coordonnées de M et de D.
\item Précise la nature du quadrilatère ABCD. Justifie.
\end{enumerate} 
\item \begin{enumerate}
\item Construis le cercle $\mathcal{(C)}$ circonscrit au quadrilatère ABCD.
\item Précise son centre, calcule son rayon et montre qu'il passe par le point O.
\end{enumerate}
\end{enumerate}
\end{exo}

\end{devoir}

\begin{td}{\matiere}{\classe}{08 mars 2020}{\prof}

\begin{exo}[4]
\begin{enumerate}
\item Calcule les nombres: $M=\frac{2}{7}-\frac{3}{7}\times(\frac{2}{3})^2$ \quad et \quad $N=(\frac{2}{3}-3)\div \frac{1}{9}+7$
\item On considère les nombres : $A=\sqrt{81}-\sqrt{108}+\sqrt{48}-\sqrt{25}$ \quad ; \quad $B=(1-\sqrt{3})^2$ \quad et \quad $C=\sqrt{A}$.
\begin{enumerate}
\item Calcule B et montre que: $A=4-2\sqrt{3}$.
\item Déduis-en une écriture simplifiée de C.
\end{enumerate} 
\end{enumerate}
\end{exo}

\vspace{0.5cm}

\begin{exo}[5]
\begin{enumerate}
\item On donne les expressions littérales suivantes: $E=2x(3x-4)-(4-3x)^2$ et $F=(9x^2-4)-(2x-3)(3x-2)$
\begin{enumerate}
\item Développe, réduis et ordonne E suivant les puissances croissantes de x.
\item Factorise E et F.
\item Résous dans R l'équation $(3x-2)(x+5)=0$.
\end{enumerate}
\item Soit Q la fraction rationnelle telle que $Q=\frac{3x(x-3)-2(x-3)}{(3x-2)(x+5)}$
\begin{enumerate}
\item Détermine la condition d'existence d'une valeur numérique de Q.
\item Simplifie Q dans cette condition.
\item Calcule la valeur exacte de Q pour $x=\sqrt{2}$.
\end{enumerate}
\end{enumerate}
\end{exo}

\vspace{0.5cm}

\begin{exo}[4]
L'unité de longueur est le centimètre. $ABC$ est un triangle rectangle en B tel que $AB=6$ et $AC=8$. M et N sont deux points tels que $\overrightarrow{AM}=\frac{1}{2}\overrightarrow{AB}$ et $\overrightarrow{AN}=\frac{1}{2}\overrightarrow{AC}$.
\begin{enumerate}
\item Trouve les distances AM et AN puis fais la figure.
\item Justifie que les droites (BC) et (MN) sont parallèles.
\item Calcule $BC$, $cos\widehat{BAC}$ et $tan\widehat{BAC}$.
\item A l'aide de la table trigonométrique, trouve l'encadrement d'ordre zéro de la mesure de l'angle $\widehat{BAC}$.
\end{enumerate}
\end{exo}

\vspace{0.5cm}

\begin{exo}[7]
Dans le plan muni d'un repère orthonormé (O, I, J), on considère les points A, B et C tel que:\\
$\overrightarrow{OA}=7\overrightarrow{OI}+\overrightarrow{OJ}$ \quad ; \quad $\overrightarrow{OB}=8\overrightarrow{OI}+4\overrightarrow{OJ}$ \quad et \quad $\overrightarrow{CO}=\overrightarrow{OI}-7\overrightarrow{OJ}$.
\begin{enumerate}
\item Place les points A, B et C dans le repère.
\item \begin{enumerate}
\item Montre que les vecteurs $\overrightarrow{AB}$ et $\overrightarrow{BC}$ sont orthogonaux.
\item Donne en justifie la nature du triangle ABC.
\end{enumerate}
\item Soient le point M milieu du segment [AC] et le point D symétrique de B par rapport à M.
\begin{enumerate}
\item Détermine les coordonnées de M et de D.
\item Précise la nature du quadrilatère ABCD. Justifie.
\end{enumerate} 
\item \begin{enumerate}
\item Construis le cercle $\mathcal{(C)}$ circonscrit au quadrilatère ABCD.
\item Précise son centre, calcule son rayon et montre qu'il passe par le point O.
\end{enumerate}
\end{enumerate}
\end{exo}
\end{td}

\end{document}
