\documentclass[12pt,a4paper]{report}
\usepackage[minitoc]{teach}
\usepackage{hyperref}
\renewcommand{\headrulewidth}{0pt}
\renewcommand{\footrulewidth}{0pt}
\renewcommand{\chaptertitlename}{Leçon}


\author{YAWO Kossi Atsu}
\newcommand{\prof}{YAWO Kossi Atsu}
\newcommand{\matiere}{MATHEMATIQUES}
\newcommand{\classe}{3$^{ème}$}
\title{Mon Cahier de Maths\\ \classe}

\setcounter{tocdepth}{1}


\newcommand{\ex}{\par \underline{Exemple:}\\}

\begin{document}
\dominitoc

\maketitle

\tableofcontents
\chapter*{Preface}
La maîtrise des notions en mathématiques par un apprenant passe avant tout par sa capacité à travailler en autonomie.

\vspace{1cm}
Cordialement YAWO Kossi Atsu.
\chapter{Opérations sur les quotients}
\section{Additionner ou soustraire des quotients}
\begin{memento}
\begin{defnslist}
\item Les formules suivantes permettent de calculer la somme ou la différence des quotients ayant le même dénominateur.\\
\fbox{$\frac{a}{d}+\frac{b}{d}=\frac{a+b}{d}$} \qquad \qquad \qquad \fbox{$\frac{a}{d}-\frac{b}{d}=\frac{a-b}{d}$}

\item Dans les situations où les quotients donnés n'ont pas le même dénominateur, on réécris les quotients pour qu'ils aient le même dénominateur sans changer leurs valeurs.
\end{defnslist}
\end{memento}

\section{Multiplier des quotients}
\begin{memento}
\fbox{$\frac{a}{b} \times \frac{c}{d}=\frac{a \times c}{b \times d}$}
\end{memento}

\section{Diviser des quotients}
\begin{memento}
Diviser deux quotients revient à multiplier le premier par l'inverse du second.\\
\fbox{$\frac{a}{b} \div \frac{c}{d}=\frac{a}{b} \times \frac{d}{c}$}
\end{memento}

\newpage

\exos
\begin{exo}
Calculer les sommes suivantes:\\
$A=\frac{4}{5}+\frac{3}{5}$ \qquad ; \qquad $B=\frac{2}{7}+\frac{1}{7}$
\end{exo}

\begin{exo}

\end{exo}

\begin{exo}

\end{exo}

\begin{exo}

\end{exo}


\chapter{Égalités de quotients}
\section{Le produit en croix}
\begin{thm}
Si $\frac{a}{b}=\frac{c}{d}$, alors $a\times d= b\times c$
\end{thm}
\section{Proportionnalité}
\begin{defn}
Dire que les nombres $x$, $y$ et $z$ sont proportionnels aux nombres $a$, $b$ et $c$ signifie que $\frac{x}{a}=\frac{y}{b}=\frac{y}{c}=k$.
\end{defn}

\begin{nb}
$k$ est appelé \textbf{le coefficient de proportionnalité}.
\end{nb}

\begin{methode}
On calcule $k$ en posant: $k=\frac{x+y+z}{a+b+c}$
\end{methode}
\exos
\begin{exo}
Trois héritiers se partagent la somme de $2 400 000$ proportionnellement à leur âges. Sachant qu'ils ont respectivement 20, 16 et 12 ans, Détermine la part de chacun.
\end{exo}

\begin{exo}
Les dimensions d'un terrain rectangulaire sont proportionnelles à 3 et 5. Sachant que le demi-périmètre vaut 560m, détermine la longueur et la largeur de ce terrain.
\end{exo}

\chapter{Puissances}
\section{Puissance à exposant négatif}
\begin{defn}
$a$ étant un nombre réel, $n$ un nombre entier naturel non nul: $a^{-n}=\frac{1}{a^{n}}$
\end{defn}
\section{Propriétés des puissances}
\begin{memento}
\begin{multicols}{2}
\begin{defenumerate}
\item $a^m \times a^n=a^{m+n}$\\
\item $a^n \times b^n=(a+b)^n$\\
\item $\frac{a^m}{a^n}=\left\{ \begin{array}{rl} a^{m-n} & \mbox{si m>n} \\ \frac{1}{a^{n-m}} & \mbox{si n>m} \end{array} \right.$\\
\item $(-a)^n=\left\{ \begin{array}{rl} a^n & \mbox{si n est pair} \\ -a^n & \mbox{si n est impair}. \end{array} \right.$
\end{defenumerate}
\end{multicols}
\end{memento}

\chapter{Développer un produit}
\section{Développements élémentaires}
\subsection{Développer une expression de la forme $a(x+y)$}
\begin{memento}
$a(x+y)=ax+ay$
\end{memento}
\subsection{Développer une expression de la forme $(a+b)(x+y)$}
\begin{memento}
$(a+b)(x+y)=ax+ay+bx+by$
\end{memento}
\subsection{Développer avec les identités remarquables}
\begin{memento}
\begin{defenumerate}
\item $(a+b)^2=a^2+2ab+b^2$
\item $(a-b)^2=a^2-2ab+b^2$
\item $(a-b)(a+b)=a^2-b^2$
\end{defenumerate}
\end{memento}
\section{Développements complexes}

\begin{nb}
Quand on doit développer une expression qui comporte plusieurs termes, on considère chaque termes comme un développement élémentaire puis on regroupe le tout après.
\end{nb}

\begin{exemple}
\begin{equation}
\begin{split}
A & =(x+2)(x+3)-2x(3x+5)+(2x-1)(x+4)\\
&=(x^2+3x+2x+6)-(6x^2+10x)+(2x^2+8x-x-4)\\
&=(x^2+5x+6)-(6x^2+10x)+(2x^2+7x-4)\\
&=x^2+5x+6-6x^2-10x+2x^2+7x-4\\
&=x^2-6x^2+2x^2+5x-10x+5x-6-4\\
&=-3x^2-6x-9
\end{split}
\end{equation}
\end{exemple}
\exos

\chapter{Factoriser une somme}
\section*{Définition}
\jar{Factoriser} une somme, c'est l'écrire sous la forme d'un \jar{produit}.

\section{Factoriser par la mise en évidence d'un facteur commun}
\begin{exemple*}
\begin{equation}
\begin{split}
F&=2x^2-5x
\end{split}
\end{equation}

\begin{equation}
\begin{split}
G&=\underline{(x-1)}(3x+5)+(2x-3)\underline{(x-1)}-4\underline{(x-1)}\\
&=(x-1)\left[(3x+5)+(2x-3)-4 \right]\\
&=(x-1)(3x+5+2x-3-4)\\
&=(x-1)(3x+2x+5-3-4)\\
G&=(x-1)(5x-2)
\end{split}
\end{equation}
\end{exemple*}

\section{Factoriser à l'aide d'une identité remarquable}
\begin{memento}
\begin{defenumerate}
\item $a^2+2ab+b^2=(a+b)^2$
\item $a^2-2ab+b^2=(a-b)^2$
\item $a^2-b^2=(a-b)(a+b)$
\end{defenumerate}
\end{memento}
\exos
\chapter{Les polynômes}
\section{Les monômes}
\begin{defn}
On appelle \jar{monôme} toute expression de la forme $ax^n$ où $a$ est nombre réel et $n$ un nombre entier naturel.
\end{defn}

\section{Les polynômes}
\begin{defn}
On appelle \jar{polynôme} toute somme algébrique de monômes.
\end{defn}

\exos

\chapter{Les fractions rationnelles}
\section{Les fractions rationnelles}
\begin{defn}
On appelle fraction rationnelle toute fraction dont le numérateur et le dénominateur sont des polynômes.
\end{defn}

\begin{exemple}
$F=\frac{x+1}{x^2-3}$ \quad ; \quad $G=\frac{(x+5)(3x-1)}{(x+4)^2}$ \quad et \quad $H=\frac{(2x-5)(x-1)}{(x+4)(x-1)}$ sont des fractions rationnelles en x.
\end{exemple}

\section{Condition d'existence d'une valeur numérique}
Une fraction rationnelle n'existe que si son dénominateur n'est pas nul (\jar{est différent de 0}).

\begin{methode}
Déterminer la condition d'existence d'une fraction rationnelle en $x$, revient à trouver toutes les valeurs de $x$ pour lesquelles son dénomateur n'est pas égal à 0.
\end{methode} 

\section{Simplifier une fraction rationnelle}
Une fraction rationnelle n'est pas toujours écrit sous sa forme la plus simple. On pourra simplifier l'écriture d'une fraction rationnelle lorsque son numérateur et son dénominateur ont un diviseur commun.

\begin{exemple}
Simplifions:\\
\begin{equation}
\begin{split}
G&=\frac{(x+1)(3x-2)}{(3x-2)(x+5)}\\
G&=\frac{x+1}{x+5}\\
\end{split}
\end{equation}
\end{exemple}

\begin{attention}
Pour simplifier une fraction rationnelle, il faut que son numérateur et son dénominateur soient sous leurs formes factorisées.
\end{attention}

\exos

\chapter{Les racines carrés}
\section{Présentation}
\begin{defn}
$a$ étant un nombre positif, la \jar{racine carrée de a} est le nombre positif dont le carré est \jar{a}.\\
\begin{center}
\includegraphics[scale=0.9,draft]{images/digramme_racine_carree.png}
\end{center}
\end{defn}

\begin{exemples}
\item $3^2=9$; $3$ étant positif, $\sqrt{9}=3$
\item $7^2=49$; $7$ étant positif, $\sqrt{49}=7$
\item $1,2^2=1,44$; $1,2$ étant positif, $\sqrt{1,44}=1,2$
\end{exemples}

\begin{nb}
La racine carrée d'un nombre $a$ n'est pas toujours un nombre décimal.
\ex
$\sqrt{5}$ n'a pas d'écriture décimale. Et par définition $(\sqrt{5})^2=5$
\end{nb}

\section{Écriture sous la forme de $a\sqrt{b}$}
\begin{prop}
Si $a$ est nombre positif, alors $\sqrt{a^2}=a$\end{prop}

\begin{methode}
La propriété précedente permet d'écrire plus simplement la racine carrée de nombres relativement grand.\\

\ex
\begin{equation}
\begin{split}
\sqrt{180} &= \sqrt{2^2 \times 3^2 \times 5}\\
&=\sqrt{2^2} \times \sqrt{3^2} \times \sqrt{5}\\
&=2 \times 3 \times \sqrt{5}\\
\sqrt{180}&=6\sqrt{5}
\end{split}
\end{equation}
\end{methode}
\exos

\chapter{Calculer avec les racines carrés}
\section{Addition et soustraction}

\section{Multiplication}
\section{Division}

\exos

\exos

\chapter{Les intervalles}
\exos

\chapter{Dénombrement}
\exos

\chapter{Approximation décimale des nombres réels}
\exos

\chapter{Les applications}
\exos

\chapter{Les applications affines}
\exos

\chapter{Sens de variation d'une application affine}
\exos

\chapter{Parler statistique}
La statistique est la branche des mathématiques qui traite des données. Elle a pour but de \jar{collecter} les informations de les \jar{classer}, puis de les \jar{interpréter}.
\par
Pour faire les statistiques, les termes suivants te seront très indispensables.

\section{Population-individu}
La \jar{population} est l'ensemble sur lequel se fait l'étude statistique.\\

\begin{exemples}
\item Les élèves d'une classe de 3ème.
\item Les articles d'un magasin.
\item Les habitant d'une région.
\end{exemples}

Chaque élément de la population considérée est appelé: un \jar{individu}.

\section{Caractère-modalité}
Le \jar{Caractère}, c'est ce qui fait l'objet de l'étude statistique.\\
En fonction du caractère étudié on peut obtenir plusieurs résultats. Ces résultats sont appelés \jar{modalités}.

\begin{exemples}
\item Si l'on considère comme population, les élèves d'une classe de $3^{ème}$, alors le caractère étudié peut être: \jar{l'âge}, \jar{le sexe}, \jar{La note obtenu à un devoir}, etc.
\item Les modalités du caractère sexe sont: \jar{masculin}, \jar{féminin}.
\end{exemples}

\section{Nature d'un caractère}
Il existe deux types de caractères.
\subsection{Caractère qualitatif}
Les modalités ne sont pas mesurables (c'est-à-dire ne sont pas des nombres).

\begin{exemples}
\item Le sexe des élèves d'une classe de $3^{ème}$.
\item L'artiste préféré par une population.
\end{exemples}

\subsection{Caractère quantitatif}
Les modalités sont pas mesurables (c'est-à-dire sont pas des nombres).

\begin{exemples}
\item La note obtenue par les élèves d'une classe de $3^{ème}$ lors d'un contrôle de Maths.
\item L'âge des élèves d'un établissement scolaire.
\end{exemples}

\newpage
\newcommand{\A}{Black-T}
\newcommand{\B}{Santrinos}
\newcommand{\C}{Almok}
\newcommand{\D}{Dian'uella}
\newcommand{\E}{Etane}

\begin{exo}
Un prof de maths se livre à une enquête auprès des élèves de son établissement afin de recueillir des informations qui lui permettront d'étudier la célébrité de certains artistes de la chanson togolaise. Voici la question qu'il leur a posé: "Parmi les artistes suivants, lequel préferez-vous: \A, \B, \C, \D, \E?\\
Les résultats obtenus sont les suivants:

\begin{remslist}
\item \A : cité 23 fois.
\item \B : cité 10 fois.
\item \C : cité 8 fois.
\item \D : cité 7 fois.
\item \E : cité 12 fois.
\end{remslist}

\begin{enumerate}
\item \begin{enumerate}
\item Quelle est la population de cette série statistique?
\item Quel est l'effectif de cette population?
\end{enumerate}
\item \begin{enumerate}
\item Quel est la caractère étudier.
\item Le caractère étudier est-il quantitatif ou qualitatif?
\item Quelles sont les modalités de ce caractère?
\end{enumerate}
\end{enumerate}
\end{exo}

\begin{exo}
Afin de choisir la couleur du maillot de l'équipe féminine de football de son école, le Directeur de l'école la \jar{Couronne d'or} se propose de faire une enquête auprès des élèves de son établissement. Les résultats obtenus sont consignés dans le tableau suivant:\\
\begin{tabular}{|p{4cm}|p{2cm}|p{2cm}|p{2cm}|}
\hline 
Couleurs & Jaune & Blanc & Vert \\ 
\hline 
Effectifs & 12 & 30 & 8 \\ 
\hline 
\end{tabular} 

\begin{enumerate}
\item \begin{enumerate}
\item Quelle est la population de cette série statistique?
\item Quel est l'effectif de cette population?
\end{enumerate}
\item \begin{enumerate}
\item Quel est la caractère étudier.
\item Quelle est sa nature?
\item Quelles sont les modalités de ce caractère?
\end{enumerate}
\end{enumerate}
\end{exo}

\begin{exo}
Voici la répartition des notes obtenues par les élèves d'une classe de $3^{ème}$ à l'issue d'un contrôle de Maths noté sur 20:\\

12 \qquad 7 \qquad 13 \qquad 18  \qquad 13 \qquad 14 \qquad 9 \qquad 9 \qquad 10 \qquad 11 \qquad 12 \qquad 7 \qquad 13 \qquad 7 \qquad 7 \qquad 7 \qquad 15 \qquad 15 \qquad 14 \qquad 14 \qquad 13 \qquad 8 \qquad 9 \qquad 7 \qquad 11.

\begin{enumerate}
\item \begin{enumerate}
\item Quelle est la population de cette série statistique?
\item Quel est l'effectif de cette population?
\end{enumerate}
\item \begin{enumerate}
\item Quel est la caractère étudier.
\item Quelle est sa nature?
\item Quelles sont les modalités de ce caractère?
\end{enumerate}
\end{enumerate}
\end{exo}

\exos

\chapter{Traitement des données} 
\begin{act}
\begin{enumerate}
\item Quelle est la proportion de fille dans votre classe?
\item Exprimer cette proportion en pourcentage.
\end{enumerate}
\end{act}


\section{La fréquence d'une modalité}
\begin{defn}
La \jar{fréquence} d'une modalité est la proportion que représente son effectif sur l'effectif total.
\end{defn}

\begin{memento}
La fréquence s'exprime généralement en pourcentage.\\
\fbox{$F=\frac{Eff. \times 100}{Eff. total}$}
\end{memento}

\section{Le mode d'une série statistique}
\begin{defn}
On appelle mode d'une série statistique, toute modalité dont l'effectif est le plus élevé.
\end{defn}
\section{Les cumules}
\begin{defns}
\item On appelle \jar{effectif cumulé} d'une modalité, la somme des effectifs des modalités inférieures ou égales à cette modalités.
\item On appelle \jar{fréquence cumulée} d'une modalité, la somme des fréquences des modalités inférieures ou égales à cette modalités.
\end{defns}

\exos

\chapter{Visualiser une série statistique par un diagramme}

\exos

\chapter{Le théorème de Thalès}
\exos

\chapter{La réciproque du théorème de Thalès}
\exos

\chapter{Le théorème de Pythagore}
\exos

\chapter{Démontrer qu'un triangle est rectangle}
\exos

\chapter{Sinus, cosinus et tangente d'un angle aigu}
\exos

\chapter{Propriétés des vecteurs}
\exos

\chapter{Repérer un point dans le plan}
\exos

\chapter{Les coordonnées d'un vecteur}
\exos

\chapter{Équations cartésiennes d'une droite}
\exos

\chapter{Les angles inscrits}
\exos

\chapter{Les pyramides et les cônes}
\exos

\end{document}
