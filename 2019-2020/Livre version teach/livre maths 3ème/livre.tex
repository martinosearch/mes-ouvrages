\documentclass[12pt,a4paper]{report}
\usepackage[minitoc]{teach}
\usepackage[utf8]{inputenc}
\usepackage[french]{babel}
\usepackage[T1]{fontenc}
\usepackage{amsmath}
\usepackage{amsfonts}
\usepackage{amssymb}
\usepackage{graphicx}
\renewcommand{\headrulewidth}{0pt}
\renewcommand{\footrulewidth}{0pt}
\renewcommand{\chaptertitlename}{Leçon}


\author{YAWO Kossi Atsu}
\newcommand{\prof}{YAWO Kossi Atsu}
\newcommand{\matiere}{MATHEMATIQUES}
\newcommand{\classe}{3$^{ème}$}
\title{Mes devoirs de Mathématiques}
\begin{document}
\dominitoc

\tableofcontents
\chapter{Sens de variation d'une application affine}

\chapter{Parler statistique}
La statistique est la branche des mathématiques qui traite des données. Elle a pour but de \jar{collecter} les informations de les \jar{classer}, puis de les \jar{interpréter}.
\par
Pour faire les statistiques, les termes suivants te seront très indispensables.

\section{Population-individu}
La \jar{population} est l'ensemble sur lequel se fait l'étude statistique.\\

\begin{exemples}
\item Les élèves d'une classe de 3ème.
\item Les articles d'un magasin.
\item Les habitant d'une région.
\end{exemples}

Chaque élément de la population considérée est appelé: un \jar{individu}.

\section{Caractère-modalité}
Le \jar{Caractère}, c'est ce qui fait l'objet de l'étude statistique.\\
En fonction du caractère étudié on peut obtenir plusieurs résultats. Ces résultats sont appelés \jar{modalités}.

\begin{exemples}
\item Si l'on considère comme population, les élèves d'une classe de $3^{ème}$, alors le caractère étudié peut être: \jar{l'âge}, \jar{le sexe}, \jar{La note obtenu à un devoir}, etc.
\item Les modalités du caractère sexe sont: \jar{masculin}, \jar{féminin}.
\end{exemples}

\section{Nature d'un caractère}
Il existe deux types de caractères.
\subsection{Caractère qualitatif}
Les modalités ne sont pas mesurables (c'est-à-dire ne sont pas des nombres).

\begin{exemples}
\item Le sexe des élèves d'une classe de $3^{ème}$.
\item L'artiste préféré par une population.
\end{exemples}

\subsection{Caractère quantitatif}
Les modalités sont pas mesurables (c'est-à-dire sont pas des nombres).

\begin{exemples}
\item La note obtenue par les élèves d'une classe de $3^{ème}$ lors d'un contrôle de Maths.
\item L'âge des élèves d'un établissement scolaire.
\end{exemples}

\newpage
\newcommand{\A}{Almok}
\newcommand{\B}{TooFan}
\newcommand{\C}{Santrinos}
\newcommand{\D}{Etan}
\newcommand{\E}{Machin}

\begin{exo}
Un prof de maths se livre à une enquête auprès des élèves de son établissement afin de recueillir des informations qui lui permettront d'étudier la célébrité de certains artistes de la chanson togolaise. Voici la question qu'il leur a posé: "Parmi les artistes suivants, lequel préferez-vous: \A, \B, \C, \D, \E?\\
Les résultats obtenus sont les suivants:

\begin{remslist}
\item \A : cité 23 fois.
\item \B : cité 10 fois.
\item \C : cité 8 fois.
\item \D : cité 7 fois.
\item \E : cité 12 fois.
\end{remslist}

\begin{enumerate}
\item \begin{enumerate}
\item Quelle est la population de cette série statistique?
\item Quel est l'effectif de cette population?
\end{enumerate}
\item \begin{enumerate}
\item Quel est la caractère étudier.
\item Le caractère étudier est-il quantitatif ou qualitatif?
\item Quelles sont les modalités de ce caractère?
\end{enumerate}
\end{enumerate}
\end{exo}

\begin{exo}
Afin de choisir la couleur du maillot de l'équipe féminine de football de son école, le Directeur de l'école la \jar{Couronne d'or} se propose de faire une enquête auprès des élèves de son établissement. Les résultats obtenus sont consignés dans le tableau suivant:\\
\begin{tabular}{|p{4cm}|p{2cm}|p{2cm}|p{2cm}|}
\hline 
Couleurs & Jaune & Blanc & Vert \\ 
\hline 
Effectifs & 12 & 30 & 8 \\ 
\hline 
\end{tabular} 

\begin{enumerate}
\item \begin{enumerate}
\item Quelle est la population de cette série statistique?
\item Quel est l'effectif de cette population?
\end{enumerate}
\item \begin{enumerate}
\item Quel est la caractère étudier.
\item Quelle est sa nature?
\item Quelles sont les modalités de ce caractère?
\end{enumerate}
\end{enumerate}
\end{exo}

\begin{exo}
Voici la répartition des notes obtenues par les élèves d'une classe de $3^{ème}$ à l'issue d'un contrôle de Maths noté sur 20:\\

12 \qquad 7 \qquad 13 \qquad 18  \qquad 13 \qquad 14 \qquad 9 \qquad 9 \qquad 10 \qquad 11 \qquad 12 \qquad 7 \qquad 13 \qquad 7 \qquad 7 \qquad 7 \qquad 15 \qquad 15 \qquad 14 \qquad 14 \qquad 13 \qquad 8 \qquad 9 \qquad 7 \qquad 11.

\begin{enumerate}
\item \begin{enumerate}
\item Quelle est la population de cette série statistique?
\item Quel est l'effectif de cette population?
\end{enumerate}
\item \begin{enumerate}
\item Quel est la caractère étudier.
\item Quelle est sa nature?
\item Quelles sont les modalités de ce caractère?
\end{enumerate}
\end{enumerate}
\end{exo}

\chapter{Étude d'un caractère qualitatif} 
\begin{act}
\begin{enumerate}
\item Quelle est la proportion de fille dans votre classe?
\item Exprimer cette proportion en pourcentage.
\end{enumerate}
\end{act}

\section{La fréquence d'une modalité}
\begin{defn}
La \jar{fréquence} d'une modalité est la proportion que représente son effectif sur l'effectif total.
\end{defn}

\section{Le mode d'une série statistique}
\section{Les cumules}
\section{Représenter une série statistique par un diagramme}

\chapter{Étude d'un caractère quantitatif}

\chapter{Les pyramides et les cônes}



\end{document}
