\documentclass[12pt,a4paper]{book}
\usepackage[minitoc]{teach}
\usepackage[utf8]{inputenc}
\usepackage[french]{babel}
\usepackage[T1]{fontenc}
\usepackage{amsmath}
\usepackage{amsfonts}
\usepackage{amssymb}
\usepackage{graphicx}
\renewcommand{\headrulewidth}{0pt}
\renewcommand{\footrulewidth}{0pt}


\author{YAWO Kossi Atsu}
\newcommand{\prof}{YAWO Kossi Atsu}
\newcommand{\matiere}{MATHEMATIQUES}
\newcommand{\classe}{3$^{ème}$}
\title{Mes devoirs de Mathématiques}
\begin{document}
\begin{examen}{BEPC-BLANC JUILLET 2020}{EPREUVE DE MATHEMATIQUES}{2}{2H}
\begin{exo}[4,5]
\begin{enumerate}
\item Calcule:\\
$A=-\frac{5}{3}\times (4+\frac{7}{5})$ \qquad ; \qquad $B=\frac{4}{5}-\frac{2}{5}:\frac{7}{15}$
\item \begin{enumerate}
\item Développe $(\sqrt{3}+1)^2$ et $(\sqrt{3}-1)^2$.
\item Déduis-en une écriture plus simple de : $A=\sqrt{4+2\sqrt{3}}$ et $B=\sqrt{4-2\sqrt{3}}$
\end{enumerate}
\item On donne les intervalles suivants: $A=]\leftarrow;-3[$ \qquad ; \qquad $B=]-5;2[$ \qquad et \qquad $C=]1;7[$
\begin{enumerate}
\item Traduis chacun de ces intervalles par une inégalité.
\item Détermine: $A\cap B$ \qquad ; \qquad $B\cup C$ \qquad et \qquad $A \cap C$
\end{enumerate}
\end{enumerate}
\end{exo}

\vspace{0.5cm}

\begin{exo}[5,5]
On donne les polynômes suivants: \\ 
$F=(x-5)(3x+4)-(2x-10)(x-1)+x-5$ \quad et \quad $G=(x-2)^2-9$.
\begin{enumerate}
\item Développe, réduis et ordonne F et G.
\item Écris F et G sous la forme de produit de facteurs du premier degré.
\item Soit la fraction rationnelle $H=\frac{(x+1)(x-5)}{(x-5)(x+7)}$.
\begin{enumerate}
\item Détermine la condition d'existence d'une valeur numérique de H.
\item Simplifie H dans cette condition d'existence.
\item Détermine la valeur x pour $H=\frac{1}{2}$.
\item Calcule la valeur numérique de H pour $x=\sqrt{3}$.
\end{enumerate}
\end{enumerate}
\end{exo}

\vspace{0.5cm}

\begin{exo}[4]
L'unité de longueur est le centimètre. $ABC$ est un triangle rectangle en B tel que $AB=12$ et $BC=5$. Fais une figure que tu compléteras au fur et à mesure.
\begin{enumerate}
\item  Marque le point D de $[AB]$ tel $AD=9$ puis trace la perpendiculaire à $(AB)$ en D; elle coupe $(AC)$ en $E$.
\item Calculer $AC$.
\competence{•}{1}
\item Que peux-tu dire des droite $(DE)$ et $(BC)$? Justifie.
\competence{•}{2}
\item Calcule les distances $AE$ et $DE$. 
\end{enumerate}
\competence{figure}{3}
\end{exo}

\vspace{0.5cm}

\begin{exo}[6]
Dans le plan muni d'un repère orthonormé (O, I, J), on considère les points A(7;1), B(8;4) et C(-1;7).
\begin{enumerate}
\item Place les points A, B et C dans le repère.
\item \begin{enumerate}
\item Calcule les distances AB, BC et AC.
\item Déduis-en la nature du triangle ABC.
\end{enumerate}
\item Soient le point M milieu du segment [AC] et le point D symétrique de B par rapport à M.
\begin{enumerate}
\item Détermine les coordonnées de M et de D.
\item Précise la nature du quadrilatère ABCD. Justifie.
\end{enumerate} 
\item \begin{enumerate}
\item Construis le cercle $\mathcal{(C)}$ circonscrit au quadrilatère ABCD.
\item Précise son centre, calcule son rayon et montre qu'il passe par le point O.
\end{enumerate}
\end{enumerate}
\end{exo}

\end{examen}

\newpage
\begin{examen}{BEPC-BLANC AOÛT 2020}{EPREUVE DE MATHEMATIQUES}{2}{2H}
\begin{exo}[5,5]
\begin{enumerate}
\item Calcule et mets les résultats sous la forme de fractions irréductibles:
$A=\frac{5}{3}-\frac{1}{3}\times\frac{9}{16}$ \qquad ; \qquad $B=\frac{3}{4}-\frac{2}{3}:\frac{18}{15}$.
\item Calcule et écris le résultat en notation scientifique: $C=\frac{8\times10^8\times1,6}{0,4\times10^{-3}}$.
\item On pose $D=\sqrt{3+2\sqrt{2}}$ et $E=\sqrt{3-2\sqrt{2}}$.
\begin{enumerate}
\item Calcule $(1+\sqrt{2})^2$ et $(1-\sqrt{2})^2$
\item En déduire la valeur la plus simple de $D$ et $E$.
\item Calcule $D+E$ ; $D-E$ et $\frac{D}{E}$.
\end{enumerate}
\item Ecris sous la forme de $a\sqrt{b}$: $C=\sqrt{12}-\sqrt{3}+\sqrt{48}$
\item Montre que $C=(\sqrt{5}+\sqrt{10})^2-10\sqrt{2}$ est un nombre entier.
\end{enumerate}
\competence{•}{4}
\end{exo}

\vspace{0.5cm}

\begin{exo}[4,5]
On condidère les expressions littérales suivantes où $x$ désigne un nombre réel:\\
$F=(x-2)(-3x+1)$ et $G=(3x-1)(2x+3)-9x^2+1$
\begin{enumerate}
\item Développe, réduis et ordonne F suivant les puissances décroissantes de x.
\item Factorise G puis résous l'équation $G=0$.
\item Soit la fraction rationnelle K définie dans $\mathbb{R}$ par $K=\frac{-3x^2+7x-2}{(3x-1)(2x+3)}$
\begin{enumerate}
\item Trouve la condition d'existence d'une valeur numérique de K.
\item Simplifie K.
\item Calcule la valeur numérique de K pour $x=\sqrt{3}$ en rendant le dénominateur rationnel.
\item Donner un encadrement de K pour $x=\sqrt{2}$ à $10^{-2}$ près sachant que $1,414<\sqrt{2}<1,415$.
\end{enumerate}
\end{enumerate}
\end{exo}

\vspace{0.5cm}

\begin{exo}[4]
L'unité de longueur est le centimètre. ABC est un triangle tel que AB=6,4; AC=4,8 et BC=8.
\begin{enumerate}
\item Démontre que le triangle ABC est rectangle.
\item Calcule l'aire du triangle ABC.
\item Soit I le milieu de [AB] et J un point du segment [BC] tel que (IJ) et (AC) soient parallèles. Démontre que J est le milieu de [BC].
\end{enumerate}
\end{exo}

\vspace{0.5cm}

\begin{exo}[6]
Dans le plan muni d'un repère orthonormé (O,I,J), on donne les points A(7;-9), B(-5;-4) et C(0;8).
\begin{enumerate}
\item Calcule les distances AB, BC et AC. Déduis-en la nature du triangle ABC.
\item Calcule les coordonnées du milieu K de [AC].
\item Soit $\mathcal{(C)}$ le cercle de diamètre [AC] et le point T(0;-9). Démontre que ce cercle passe par B et T.
\item Soit D le symétrique de B par rapport à K. Quelle est la nature du quadrilatère ABCD.
\item Soit $(\Delta)$ la parallèle à l'axe des ordonnées passant par le point B. Cette droite recoupe le cercle $\mathcal{(C)}$ en un point E. Montre que les droites (DE) et (BE) sont perpendiculaires.
\end{enumerate}
\end{exo}

\end{examen}

\end{document}
