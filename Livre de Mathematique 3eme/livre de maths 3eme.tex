\documentclass[12pt,a4paper]{book}
\usepackage[minitoc]{teach}
\title{Mon Cahier de Mathématiques}
\author{Kossi Atsu YAWO}
\begin{document}
\maketitle
\dominitoc % <-- Ne pas oublier cette ligne !
\tableofcontents
\chapter{Les nombres réels}
\section{Les nombres rationnels}
\subsection{Rappels sur les nombres rationnels}
\section{Les nombres irrationnels}
\subsection{Présentation}
\subsection{Les racines carrées}

\chapter{Quotients}
\section{Présentation}
\begin{defn}
On appelle quotient, toute écriture de la forme $\frac{a}{b}$, ou a et b sont des nombres réels.
\end{defn}

\begin{exemple}
$\frac{2}{3}$ \qquad ; \qquad $\frac{-2}{3}$ \qquad  et \qquad $\frac{2,8}{3}$ \qquad  sont des quotients
\end{exemple}

\section{Égalité de quotients}
\begin{thm}
$\frac{a}{b} = \frac{c}{d}$ équivaut à $a\times d=b \times c$
\end{thm}
\section{Opération sur les quotients}

\chapter{Calcul littéral}
\section{Les polynômes}
\subsection{Présentation}
\subsection{Différentes écritures d'un polynôme}
\subsubsection{Développement}
\subsubsection{Factorisation}
\section{Les fractions rationnelles}
\chapter{Les équations dans $\mathbb{R}$}
\chapter{Les intervalles}
\chapter{Les inéquations dans $\mathbb{R}$}
\chapter{Comparaison des nombres réels}
\chapter{Approximation décimale d'un nombre réel}
\chapter{Dénombrement}
\chapter{Les équations dans $\mathbb{R} \times \mathbb{R}$}
\chapter{Les inéquations dans $\mathbb{R} \times \mathbb{R}$}
\chapter{Les applications affines}
\chapter{La statistique}
\chapter{Le triangle rectangle}
\chapter{Le théorème de Thalès}
\chapter{Les vecteurs}
\chapter{Les coordonnées d'un vecteur}
\chapter{Équation cartésienne d'une droite}
\chapter{Les angles inscrits}
\chapter{Les pyramides et les cônes}
\end{document}
\end{document}